\chapter{Conclusion}
In this project, we implemented the novel ANN method by Professor \textit{Savchenko} with the
maximal likelihood search of the next reference instance
in order to improve the performance of the NN-based image
recognition methods. We analyzed its modification P-ML-ANN to
make this approach suitable for practical applications. Several pivots
are chosen at the preprocessing stage of the P-ML-ANN
method. Hence, the main cycle of the search procedure
becomes as fast as the brute force. Although this approach does
not achieve the lowest average number of computed distances as
the ML-ANN, it is more computationally efficient in most cases.
Our modification requires only linear memory space to store the
distances between pivots and all reference images. An obvious
drawback of this procedure is the need for a proper choice of the
number of pivots.e Invariant Feature Transform.