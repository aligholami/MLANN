\chapter{Introduction}

In this section we'll take a brief look at the recent researches on \textit{Template Matching} task, its variants and the reliability of the models proposed in this task.
\section{Background and Recent Research}

\subsection{Feature Based Approach}
A featured-based approach is appropriate when both reference
and template images contain more correspondence
with respect to features and control points. In this case,
features include points, curves, or a surface model to perform
template matching. In this category, the final goal is to locate
the pair-wise connections between the target or so-called
reference and the template image using spatial relations
or features. In this approach, spatial relations, invariant
descriptors, pyramids, wavelets and relaxation methods play
an important role in extracting matching measures\cite{abstract}.

\subsection{Area Based Approach}
Area-based methods, which are usually known as correlation
methods or template matching, were developed for the
first time by Fonseca et al. [6] and are based on a combined
algorithm of feature detection and feature matching. This
method functions very well when the templates have no
strong features with an image, since they operate directly on
the pixel values. Matches are measured using the intensity
values of both image and template. The matching scores
are extracted by calculating squared differences in fixed
intensities, correction-based methods, optimization methods
and mutual information\cite{intro1}.

\subsection{Naive Template Matching}
Nave template matching is one of the basic methods of
extracting a given which is identical to the template from
the image target. In this approach, with or without scaling
(usually without scaling), the target image is scanned by the
template, and the similarity measures are calculated. Finally,
the positions with the strongest similarities are identified as
potential pattern positions\cite{abstract}. 

\subsection{Image Correlation Matching}
In this classic template matching method, the similarity
metric between the target and the template is measured.
Unlike the naive template matching algorithm, the target and the template might have different image intensities or
noise levels. However, those images must be aligned. The
similarity metric used in this approach is based on the
correlation between the target and the template\cite{intro2}.

\subsection{Sequential Similarity Detection Algorithms}
Sequential similarity detection algorithms (SSDAs) are
a more efficient alternative to correlation-based methods,
including matched filters for translational registration. The
measure of match is indirectly calculated based on an error
for corresponding pixels in f and g in the images under
comparison at any stage of the registration process\cite{intro3}.

\section{Motivation}
There are dozen optimized implementation of \textit{Template Matching} techniques, but the main motivation for me to do so, was the ability to model and code a real-life problem from scratch. I've also improved my CUDA and C++ programming skills with this project. Also, there are multiple math concepts covered in the \textit{Fast Fourier Transform} section which understanding them can bring an enhanced point of view in image and signal processing tasks.