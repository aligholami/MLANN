\chapter{Introduction}

\section{Motivation}
One of the well-known issues of vision systems building, is a processing of
large databases. Unfortunately, nearest neighbor (NN) rule and exhaustive
search usually cannot be implemented in real-time applications\cite{motiv1}. To overcome these problems, we'll analyze some of the purposed methods in this this topic.

\section{Recent Methods of Image Recognition in Large Scale Databases}
In this section we'll take a brief look at the recent researches on image recognition systems and their improvements for large scale databases.

\subsection{Approximate Nearest Neighbor Shape Indexing}
This method relies on the rapid recovery of the nearest neighbors from the index. In high-dimensional databases, standard \textit{k}-d tree search often performs poorly.  having to examine a large fraction of the points in the space to find the exact nearest neighbor. However, a variant of this search which efficiently finds approximate neighbors will be used to limit the search time. The algorithm, which we have called Best
Bin First (BBF) search, finds the nearest neighbor for a
large fraction of queries, and finds a very good neighbor
the remaining times \cite{motiv2}.
	
\subsection{Directed Enumeration Method}
Directed enumeration method is proposed to improve image recognition performance. The method is applied with similarity measures which do not met metric properties. This method increases performance in 3 to 12 times in comparison with nearest neighbor. Recognition speed using \textit{DEM} is increased when many neighbors are located at similar distances \cite{motiv3}.


\subsection{Directed Enumeration Alternatives Modification}
In this method, a new modification of the method of directed alternatives’
enumeration using the Kullback Leibler discrimination information is proposed
for half-tone image recognition. Results of an experimental study in
the problem of face images recognition with a large database are presented.
It is shown that the proposed modification is characterized by increased
speed of image recognition (5-10 times vs exhaustive search) \cite{motiv4}.
