\chapter{Problem Definition}

In this chapter, we'll take a close look at the core algorithm of \textit{MLANN} method. The next is dedicated to the implementation procedure of the algorithm.

\subsection{MLANN; General Idea} 
Despite the most of known fast approximate NN algorithms, the proposed method is not heuristic. The joint probabilistic densities (likelihoods) of the distances to previously checked reference objects are estimated for each class. The next reference instance is selected from the class with the maximal likelihood. To deal with the quadratic memory requirement of this approach, the author has proposed its modification, which processes the distances from all instances to a small set of pivots chosen with the farthest-first traversal. Experimental study in face recognition with the histograms of oriented gradients and the deep neural network-based image features shows that the proposed method is much faster than the known approximate NN algorithms for medium databases.

