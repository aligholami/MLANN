\documentclass[journal, a4paper]{IEEEtran}

% some very useful LaTeX packages include:

%\usepackage{cite}      % Written by Donald Arseneau
                        % V1.6 and later of IEEEtran pre-defines the format
                        % of the cite.sty package \cite{} output to follow
                        % that of IEEE. Loading the cite package will
                        % result in citation numbers being automatically
                        % sorted and properly "ranged". i.e.,
                        % [1], [9], [2], [7], [5], [6]
                        % (without using cite.sty)
                        % will become:
                        % [1], [2], [5]--[7], [9] (using cite.sty)
                        % cite.sty's \cite will automatically add leading
                        % space, if needed. Use cite.sty's noadjust option
                        % (cite.sty V3.8 and later) if you want to turn this
                        % off. cite.sty is already installed on most LaTeX
                        % systems. The latest version can be obtained at:
                        % http://www.ctan.org/tex-archive/macros/latex/contrib/supported/cite/

\usepackage{graphicx}   % Written by David Carlisle and Sebastian Rahtz
                        % Required if you want graphics, photos, etc.
                        % graphicx.sty is already installed on most LaTeX
                        % systems. The latest version and documentation can
                        % be obtained at:
                        % http://www.ctan.org/tex-archive/macros/latex/required/graphics/
                        % Another good source of documentation is "Using
                        % Imported Graphics in LaTeX2e" by Keith Reckdahl
                        % which can be found as esplatex.ps and epslatex.pdf
                        % at: http://www.ctan.org/tex-archive/info/

%\usepackage{psfrag}    % Written by Craig Barratt, Michael C. Grant,
                        % and David Carlisle
                        % This package allows you to substitute LaTeX
                        % commands for text in imported EPS graphic files.
                        % In this way, LaTeX symbols can be placed into
                        % graphics that have been generated by other
                        % applications. You must use latex->dvips->ps2pdf
                        % workflow (not direct pdf output from pdflatex) if
                        % you wish to use this capability because it works
                        % via some PostScript tricks. Alternatively, the
                        % graphics could be processed as separate files via
                        % psfrag and dvips, then converted to PDF for
                        % inclusion in the main file which uses pdflatex.
                        % Docs are in "The PSfrag System" by Michael C. Grant
                        % and David Carlisle. There is also some information
                        % about using psfrag in "Using Imported Graphics in
                        % LaTeX2e" by Keith Reckdahl which documents the
                        % graphicx package (see above). The psfrag package
                        % and documentation can be obtained at:
                        % http://www.ctan.org/tex-archive/macros/latex/contrib/supported/psfrag/

%\usepackage{subfigure} % Written by Steven Douglas Cochran
                        % This package makes it easy to put subfigures
                        % in your figures. i.e., "figure 1a and 1b"
                        % Docs are in "Using Imported Graphics in LaTeX2e"
                        % by Keith Reckdahl which also documents the graphicx
                        % package (see above). subfigure.sty is already
                        % installed on most LaTeX systems. The latest version
                        % and documentation can be obtained at:
                        % http://www.ctan.org/tex-archive/macros/latex/contrib/supported/subfigure/

\usepackage{url}        % Written by Donald Arseneau
                        % Provides better support for handling and breaking
                        % URLs. url.sty is already installed on most LaTeX
                        % systems. The latest version can be obtained at:
                        % http://www.ctan.org/tex-archive/macros/latex/contrib/other/misc/
                        % Read the url.sty source comments for usage information.

%\usepackage{stfloats}  % Written by Sigitas Tolusis
                        % Gives LaTeX2e the ability to do double column
                        % floats at the bottom of the page as well as the top.
                        % (e.g., "\begin{figure*}[!b]" is not normally
                        % possible in LaTeX2e). This is an invasive package
                        % which rewrites many portions of the LaTeX2e output
                        % routines. It may not work with other packages that
                        % modify the LaTeX2e output routine and/or with other
                        % versions of LaTeX. The latest version and
                        % documentation can be obtained at:
                        % http://www.ctan.org/tex-archive/macros/latex/contrib/supported/sttools/
                        % Documentation is contained in the stfloats.sty
                        % comments as well as in the presfull.pdf file.
                        % Do not use the stfloats baselinefloat ability as
                        % IEEE does not allow \baselineskip to stretch.
                        % Authors submitting work to the IEEE should note
                        % that IEEE rarely uses double column equations and
                        % that authors should try to avoid such use.
                        % Do not be tempted to use the cuted.sty or
                        % midfloat.sty package (by the same author) as IEEE
                        % does not format its papers in such ways.

\usepackage{amsmath}    % From the American Mathematical Society
                        % A popular package that provides many helpful commands
                        % for dealing with mathematics. Note that the AMSmath
                        % package sets \interdisplaylinepenalty to 10000 thus
                        % preventing page breaks from occurring within multiline
                        % equations. Use:
%\interdisplaylinepenalty=2500
                        % after loading amsmath to restore such page breaks
                        % as IEEEtran.cls normally does. amsmath.sty is already
                        % installed on most LaTeX systems. The latest version
                        % and documentation can be obtained at:
                        % http://www.ctan.org/tex-archive/macros/latex/required/amslatex/math/
\usepackage{amsmath}
\DeclareMathOperator*{\argmax}{arg\,max}
\DeclareMathOperator*{\argmin}{arg\,min}

\usepackage{url}
\usepackage{color,hyperref}
\definecolor{darkblue}{rgb}{0.0,0.0,0.55}
\hypersetup{colorlinks,breaklinks,
	linkcolor=darkblue,urlcolor=darkblue,
	anchorcolor=darkblue,citecolor=darkblue}
% Other popular packages for formatting tables and equations include:

%\usepackage{array}
% Frank Mittelbach's and David Carlisle's array.sty which improves the
% LaTeX2e array and tabular environments to provide better appearances and
% additional user controls. array.sty is already installed on most systems.
% The latest version and documentation can be obtained at:
% http://www.ctan.org/tex-archive/macros/latex/required/tools/

% V1.6 of IEEEtran contains the IEEEeqnarray family of commands that can
% be used to generate multiline equations as well as matrices, tables, etc.

% Also of notable interest:
% Scott Pakin's eqparbox package for creating (automatically sized) equal
% width boxes. Available:
% http://www.ctan.org/tex-archive/macros/latex/contrib/supported/eqparbox/

% *** Do not adjust lengths that control margins, column widths, etc. ***
% *** Do not use packages that alter fonts (such as pslatex).         ***
% There should be no need to do such things with IEEEtran.cls V1.6 and later.


% Your document starts here!
\begin{document}

% Define document title and author
	\title{MLANN; Maximum Likelihood Approximate Nearest Neighbor in Real-time Image Recognition}
	\author{\textsc{Ali Gholami\\ Department of Computer Engineering \& Information Technology\\ Amirkabir University of Technology}\\\textit{\url{https://aligholamee.github.io}\\\url{aligholami7596@gmail.com}}
	\thanks{Advisor: Professor Mohammad Rahmati}}
	\markboth{Statistical Pattern Recognition Final Project -- Advancement Report -- May 2, 2018}{}
	\maketitle

% Write abstract here
\begin{abstract}
	In this report, the core algorithm of the \textit{MLANN} is observed. We'll introduce the dataset which is going to be used as our train data. We'll also estimate the computational requirements for this project.
\end{abstract}

% Each section begins with a \section{title} command
\section{Introduction}
	% \PARstart{}{} creates a tall first letter for this first paragraph
	\PARstart{f}{or} the task of statistical image recognition, the unknown densities of each class should be estimated at the first step. Suppose we have \textit{R} reference images as $r\ \epsilon\ \{1, 2, 3, ..., R\}$. Statistical image recognition can be reduced to computing the distances between the input image $X$ and the reference images $X_r$ which is formally illustrated below:
	$$
		W_v: v = \text{arg}\,\min\limits_{r} \rho(X, X_r)
	$$
	$W_v$ is the actual class of the input image.
	
\section{Initial Intuition}
The core implementation of this algorithm is where the reference images are being selected. In the nearest neighbor method, the whole database is brute forced to find the nearest image to the input image. But in this method, there are conceptually different approaches that can speed up finding the best match for the input image. This method conducts the Best-bin-first kd-trees (An extension of classic kd-trees) to create a priority queue of the reference images. Next, the highest priority item $X_i$ is pulled from the queue and the set of reference images $X_i^{(M)}$ is determined by the following expression:
$$
	(\forall X_k \epsilon X_i^{(M)})(\forall X_j \not\epsilon X_i^{(M)})
$$
$$
	|\rho_{i, k} - \rho(X, X_k)| \leq |\rho_{i, j} - \rho(X, X_j)|
$$
Note that $\rho_{i, k}$ are the elements of the distance matrix $P = [\rho_{i, j}]$ which is computed in the preliminarily step.

\section{Assumptions}
Here is the main assumptions of the algorithm we'll discuss below.
\begin{enumerate}
	\item The reference images from different
	classes are independent.
	
	\item The probability distributions of the
	feature vectors from the same class are identical.
\end{enumerate}
\section{Core Algorithm} 
Let the reference images $X_{r_1}, X_{r_2}, ..., X_{r_k}$ have been checked before the k-th step, i.e. the distances $\rho(X, X_{r_1}), \rho(X, X_{r_2}), ... \rho(X, X_{r_k})$ are computed. The next reference image is selected using the following formula:
$$
	r_{k+1} = \underset{v \epsilon \{1, ..., R\} - \{r_1, ..., r_k\}}{argmax}(p_v. \prod_{i = 1}^{k} f(\rho(X, X_{r_i}) | W_v))
$$
where $\prod_{i = 1}^{k} f(\rho(X, X_{r_i}) | W_v)$ is the conditional density of the distance $\rho(X, X_{r_i})$ if the hypothesis $W_v$ is true. If the prior probabilities of all classes are not identical, then the next instance will be selected from the majority classes with high probability. 

\section{Datasets \& Implementation Progress}
The initial platform for the face recognition is implemented using \textit{OpenCV} and \textit{Tensorflow}. Faces are detected using the \textit{MTCNN}\cite{HOP96} method which is an extension of the implementation of \textit{David Sandberg} (FaceNet's Contributor). The features are extracted in real time using \textit{ResNet V1}\cite{HOP97}. It is planned to generate a database of extracted features using \textit{13000} images of the \textit{LFW} dataset which consists of labeled faces in approximately 1600 categories. Note that the current implementation of the project, which we call it \textit{BioFace} from this day forward, uses the naive nearest neighbor brute force search to find the best match of the input image among the reference images in database.
% Now we need a bibliography:
\begin{thebibliography}{5}
	
	\bibitem{HOP96} % Transaction paper
	Zhang, Kaipeng, et al. "Joint face detection and alignment using multitask cascaded convolutional networks." IEEE Signal Processing Letters 23.10 (2016): 1499-1503.

	\bibitem{HOP97} % Transaction paper
	He, Kaiming, et al. "Deep residual learning for image recognition." Proceedings of the IEEE conference on computer vision and pattern recognition. 2016.
\end{thebibliography}


% Your document ends here!
\end{document}